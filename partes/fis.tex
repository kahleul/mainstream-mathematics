%!TEX root = main.tex

\chapter{Capítulo 2 Moysés}

\section{Movimento Retilíneo}

\begin{defi}
    Considere um movimento retilíneo de equação horária $x = x(t)$.
        \begin{enumerate}[leftmargin=*, align=left, label=\textbf{(\alph*)}]
            \item A \textit{velocidade média} $\bar{v}_{t_1 \to t_2}$ entre os instantes $t_1$ e $t_2$ é definida como
                \[
                    \bar{v}_{t_1 \to t_2} := \dfrac{x(t_2) - x(t_1)}{t_2 - t_1}.
                \]
            \item A \textit{velocidade} $v(t)$ num instante $t$ qualquer é definida como
                \[
                    v(t) := \dfrac{dx}{dt} = x'(t) =  \lim_{\Delta{t} \to 0} \dfrac{x(t + \Delta{t}) - x(t)}{\Delta{t}}.
                \]
        \end{enumerate}
\end{defi}

\begin{prop}
    Considere um movimento retilíneo de lei horária $x = x(t)$ e velocidade $v = v(t)$. A posição $x(t)$ no instante $t$ depois de um instante $t_0$ é
        \[
            x(t) = x(t_0) + \int_{t_0}^{t} v(t') \, dt'.
        \]
\end{prop}

\begin{proof}
    Segue imediatamente do TFC. \itemproof
\end{proof}

\begin{defi}
    Considere um movimento retilíneo de velocidade $v = v(t)$.
        \begin{enumerate}[leftmargin=*, align=left, label=\textbf{(\alph*)}]
            \item A \textit{aceleração média} $\bar{a}_{t_1 \to t_2}$ entre os instantes $t_1$ e $t_2$ é definida como
                \[
                    \bar{a}_{t_1 \to t_2} := \dfrac{v(t_2) - v(t_1)}{t_2 - t_1}.
                \]
            \item A \textit{aceleração} $a(t)$ num instante $t$ qualquer é definida como
                \[
                    a(t) := \dfrac{dv}{dt} = v'(t) = \lim_{\Delta{t} \to 0} \dfrac{v(t + \Delta{t}) - v(t)}{\Delta{t}}.
                \]
        \end{enumerate}
\end{defi}

\begin{prop}
    Considere um movimento retilíneo de velocidade $v = v(t)$ e aceleração $a = a(t)$. A velocidade $v(t)$ num instante $t$ qualquer depois de um instante $t_0$ é
        \[
            v(t) = v(t_0) + \int_{t_0}^{t} a(t') \, dt'.
        \]
\end{prop}

\begin{proof}
    Segue imediatamente do TFC. \itemproof
\end{proof}

\subsection{Movimento Retilíneo Uniforme}

\begin{defi}
    Um movimento retilíneo é \textit{uniforme} (MRU) se sua velocidade é constante.
\end{defi}

\begin{prop}
    Todo MRU tem lei horária linear.
\end{prop}

\begin{proof}
    O deslocamento $\Delta{x}$ entre os instantes $t_0$ e $t$ é
        \[
                x(t) - x(t_0) = \int_{t_0}^{t} v(t) \, dt.
        \]
    Como a velocidade é constante (por definição), temos $v(t) = v$, donde
        \[
            x(t) - x(t_0) = \int_{t_0}^{t} v \, dt \Rightarrow x(t) = x(t_0) + v(t-t_0),
        \]
    como havíamos afirmado. \itemproof
\end{proof}

\begin{teo}
    A velocidade média de um MRU coincide com a velocidade.
\end{teo}

\begin{proof}
    \leavevmode
        \begin{enumerate}[leftmargin=*, align=left, label=\textbf{(\alph*)}]
            \item Sendo $x(t) = x_0 + v(t-t_0)$ a lei horária do MRU, onde $x_0 = x(t_0)$, temos
                \begin{align*}
                    \bar{v}_{t_1 \to t_2} &= \dfrac{x(t_2) - x(t_1)}{t_2 - t_1} \\
                    &= \dfrac{[x_0 + v(t_2 - t_0)] - [x_0 + v(t_1 - t_0)]}{t_2 - t_1} \\
                    &= \dfrac{v(t_2 - t_1)}{t_2 - t_1} = v
                \end{align*}
            para quaisquer instantes $t_1$ e $t_2$. \itemproof
        \end{enumerate}
\end{proof}

\subsection{Movimento Retilíneo Uniformemente Acelerado}

\begin{defi}
    \leavevmode
        \begin{enumerate}[leftmargin=*, align=left, label=\textbf{(\alph*)}]
            \item Um movimento retilíneo é \textit{acelerado} se não é uniforme.
            \item Um movimento retilíneo é \textit{uniformemente acelerado} se sua aceleração é constante e não nula.
        \end{enumerate}
\end{defi}

\begin{prop}
    \leavevmode
        \begin{enumerate}[leftmargin=*, align=left, label=\textbf{(\alph*)}]
            \item Todo MRUA tem velocidade linear.
            \item Todo MRUA tem equação horária quadrática.
        \end{enumerate}
\end{prop}

\begin{proof}
    \leavevmode
        \begin{enumerate}[leftmargin=*, align=left, label=\textbf{(\alph*)}]
            \item A variação de velocidade $\Delta{v}$ entre os instantes $t_0$ e $t$ é
                \[
                    v(t) - v(t_0) = \int_{t_0}^{t} a(t) \, dt.
                \]
            Como a aceleração é constante (por definição), temos $a(t) = a$, donde
                \[
                    v(t) - v(t_0) = \int_{t_0}^{t} a \, dt \Rightarrow v(t) = v(t_0) + a(t-t_0),
                \]
            como havíamos afirmado. \itemproof
            \item O deslocamento $\Delta{x}$ entre os instantes $t_0$ e $t$ é
                \[
                    x(t) - x(t_0) = \int_{t_0}^{t} v(t') \, dt'.
                \]
            Com $v(t') = v(t_0) + a(t'-t_0)$, integrando obtemos
                \[
                    \int_{t_0}^{t} v(t') \, dt' = \int_{t_0}^{t} v(t_0) + a(t'-t_0) \, dt' = v(t_0)(t-t_0) + \dfrac{a(t-t_0)^2}{2},
                \]
            donde por fim segue que
                \[
                    x(t) = x(t_0) + v(t_0)(t-t_0) + \dfrac{a(t-t_0)^2}{2}, 
                \]
            como havíamos afirmado. \itemproof
        \end{enumerate}
\end{proof}

\begin{teo}
    Num MRUA, valem as seguintes afirmações.
        \begin{enumerate}[leftmargin=*, align=left, label=\textbf{(\alph*)}]
            \item A velocidade média $\bar{v}_{t_0 \to t}$ entre instantes $t_0$ e $t$ é a média entre as velocidades nesses instantes:
                \[
                    \bar{v}_{t_0 \to t} = \dfrac{v(t_0) + v(t)}{2} 
                \]
            \item (Torricelli) A posição $x(t)$ num instante $t$ qualquer depois de um instante $t_0$ pode ser calculada em função de $v(t)$, $v(t_0)$ e $x(t_0)$:
                \[
                    x(t) = x(t_0) + \dfrac{v^2(t)-v^2(t_0)}{2a}.
                \]
        \end{enumerate}
\end{teo}

\begin{proof}
    \leavevmode 
        \begin{enumerate}[leftmargin=*, align=left, label=\textbf{(\alph*)}]
            \item 
        \end{enumerate}
\end{proof}


